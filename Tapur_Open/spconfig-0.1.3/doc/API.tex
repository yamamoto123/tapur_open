\documentclass[10pt]{article}

\usepackage{vmargin}
\setpapersize{A4}
% {left}{top}{right}{bottom}{headheight}{headsep}{footheight}{footskip}
\setmarginsrb{1.5in}{1in}{1in}{1.5in}{0cm}{0cm}{0pt}{0pt}

\title{spConfig-0.1.3 API Reference}
\author{Christopher Wright, cwright@softpixel.com\\
	Steve Mokris, smokris@softpixel.com}
\date{\today}
\begin{document}

\maketitle

\tableofcontents

\section{Introduction}

To visualize the way configuration files work in spConfig, one simply needs to
think of a hierarchical filesystem.  Configuration files are arranged in a
tree-like structure, composed of `nodes' (analogous to directories),
`attributes' (files), and `text' (which is a sort of metadata).  Additionally,
commands such as `notices' can be inserted into the config tree. There are
plans for simple logic commands and an `include' operation in the future.

The syntax is similar to XML, and the parser should be able to grok most XML
files.  We will perform a more extensive review of this in the future.  For a
fairly complete example and documentation of the syntax, see demo/demo.conf.




\pagebreak

\section{Structure}

The base of any config tree is a particular node, known as the \textbf{root
node}.  Typical operation involves using the library to read from a config file
(on disk or in memory) into a memory structure, then using the library-provided
routines to glean information from this structure.  It is also possible to use
the library-provided routines to construct such a structure in its entirety. 
In the future, it will be possible to write out these memory structures to disk
in our config file format.

\subsection{struct spConfigNode}

\begin{verbatim}typedef struct spConfigNodeS
{
        char                    *name;
        char                    confFlag;
        char                    *text;
        
        int                     numAttributes;
        int                     numChildren;
        spConfigAttribute       **attributes;
        struct spConfigNodeS    **children;
	
        struct spConfigNodeS    *parent;
        spConfigMessage         *message;
        void*(*callback)(struct spConfigNodeS*node,char*string);
} spConfigNode;\end{verbatim}

\begin{itemize}
	\item[name] is obviously the name of the node. When reading in a configuration file from disk, the root node's name points to NULL, but
this is not a requirement for structures created in memory (more on this below).
	\item[text] stores the text that happened to be located in the node, between the opening tag and closing tag, if any.  It does not include text within child nodes.
	\item[numAttributes] number of attributes this node has
	\item[numChildren] number of children this node has
	\item[**attributes] one-dimensional array of attributes
	\item[**children] one-dimensional array of child nodes.
	\item[parent] is a pointer to the parent node of this node, if there is any (NULL if root node). 
Unfortunately, sharing a node between two parents will result in a pointer to only one.
	\item[message] is the beginning of a linked list of messages attached to this node.
	\item[confFlag] can have any of the following set:
\begin{verbatim}
SP_CONFIG_NODE_ALLOW_TEXT   - this node accepts text if parsed
SP_CONFIG_NODE_ALLOW_ATTRIBUTES - this node accepts additional attributes
SP_CONFIG_NODE_ALLOW_NODES  - this node accepts additional child nodes
\end{verbatim}

These flag values are not yet implemented, however.
	\item[callback] is a function used to handle include files.  Because we don't support
include files just yet, this field is presently useless.
\end{itemize}



\subsection{struct spConfigAttribute}

\noindent\textbf{attributes} are name/value pairs defined within the node begin tag.
\begin{verbatim}
typedef struct
{
        char    *name;
        char    type;
        union
        {
                int i;
                double d;
                char *string;
        }       value;
} spConfigAttribute;
\end{verbatim}

\begin{itemize}
	\item[name] stores the attributes name (this should always be set).
	\item[type] determines which type of data this attribute holds, if any.\\
\begin{verbatim}SP_CONFIG_VALUE_NONE    - no values are associated with this attribute
SP_CONFIG_VALUE_INT     - the value is an int (use value.i)
SP_CONFIG_VALUE_FLOAT   - the value is a double (use value.d)
SP_CONFIG_VALUE_STRING  - the value is a string (use value.string);\end{verbatim}
\end{itemize}


\subsection{struct spConfigMessage}

\noindent And finally, we have \textbf{messages}.\\
\begin{verbatim}
typedef struct spConfigMessageS
{
        int     line,col;
        char    type;
        char    *text;
        char    *filename;
        struct spConfigMessageS *next;
} spConfigMessage;
\end{verbatim}

\begin{itemize}
	\item[line] number (of input file/string) upon which this message occurred
	\item[col] column of message
	\item[type] determines the severity of the message:\\
\begin{verbatim}
SP_CONFIG_MESSAGE_INFO          - information, not a problem
SP_CONFIG_MESSAGE_NOTICE        - small trivial warning, nothing too bad
SP_CONFIG_MESSAGE_WARNING       - something may have been misparsed
                                  but overall we're still on track.
SP_CONFIG_MESSAGE_ERROR         - something is very wrong, parsing may 
                                  be wrong after this point.
\end{verbatim}

	\item[text] stores the actual message.
	\item[filename] stores which file the message
originated in (currently useless, we don't support include files yet).
	\item[next] stores a pointer to the next message, or NULL of there
are no more messages.
\end{itemize}









\pagebreak

\section{API}

\subsection{Allocation / Duplication / Deallocation}

\subsubsection*{New}
\noindent\begin{tabular}{l l l @{}}
spConfigNode&\textbf{*spConfigNodeNew}&();\\
spConfigAttribute&\textbf{*spConfigAttributeNew}&();\\
spConfigMessage&\textbf{*spConfigMessageNew}&();\\
\end{tabular}

\noindent These functions return a newly allocated and initialized item, or
NULL on error (out of memory).

\subsubsection*{Clone}
\noindent\begin{tabular}{l l l @{}}
spConfigNode&\textbf{*spConfigNodeClone}&(spConfigNode *node);\\
spConfigAttribute&\textbf{*spConfigAttributeClone}&(spConfigAttribute *att);\\
spConfigMessage&\textbf{*spConfigMessageClone}&(spConfigMessage *msg);\\
\end{tabular}

\noindent These functions return a complete copy of the input item.  Strings
are copied, child nodes and values are copied, and any other pointers are
duplicated, not shared.  Thus it is safe to clone a tree and hack on one.  The
other will be preserved.

\subsubsection*{Copy}
\noindent\begin{tabular}{l l l @{}}
spConfigNode&\textbf{*spConfigNodeCopy}&(spConfigNode *node);\\
spConfigAttribute&\textbf{*spConfigAttributeCopy}&(spConfigAttribute *att);\\
spConfigMessage&\textbf{*spConfigMessageCopy}&(spConfigMessage *msg);\\
\end{tabular}

\noindent These functions copy the values of the input item, returning a newly
allocated  item with the same values. Pointers are shared, so copying a tree
and hacking on one  will hack up the other as well.  Very hard to cleanly
destruct (pointers could already  be freed), these probably shouldn't be used
unless you know what you're doing.

\subsubsection*{Free}
\noindent\begin{tabular}{l l l @{}}
void&\textbf{spConfigNodeFree}&(spConfigNode *node);\\
void&\textbf{spConfigAttributeFree}&(spConfigAttribute *att);\\
void&\textbf{spConfigMessageFree}&(spConfigMessage *msg);\\
\end{tabular}

\noindent These functions free the memory used by the item passed to them. 
Messages and nodes have their children freed as well.




\subsection{Node Management}

\subsubsection*{AddChild}
\noindent\begin{tabular}{l l l @{}}
int&\textbf{spConfigNodeAddChild}&(spConfigNode *parent,spConfigNode *child);\\
\end{tabular}

\noindent This function adds node \textbf{child} to the list of child nodes in
\textbf{parent}.\\ Returns 0 on success, else error.

\subsubsection*{AddAttribute}
\noindent\begin{tabular}{l l l @{}}
int&\textbf{spConfigNodeAddAttribute}&(spConfigNode *parent,spConfigAttribute *att);\\
\end{tabular}

\noindent This adds value \textbf{att} to the list of attributes in
\textbf{parent}.\\ Returns 0 on success, else error.

\subsubsection*{AddAttribute*}
\noindent\begin{tabular}{l l l @{}}
int&\textbf{spConfigNodeAddAttributeInt}&(spConfigNode*parent,char*name,int i);\\
int&\textbf{spConfigNodeAddAttributeDouble}&(spConfigNode*parent,char*name,double d);\\
int&\textbf{spConfigNodeAddAttributeString}&(spConfigNode*parent,char*name,char*text);\\
\end{tabular}

\noindent These functions add an attribute named \textbf{name}, with a value of
the item passed,  to the list of attributes in \textbf{parent}.\\ They return 0
on success, else error.

\subsubsection*{NodeName}
\noindent\begin{tabular}{l l l @{}}
int&\textbf{spConfigNodeName}&(spConfigNode *node,char *name);\\
\end{tabular}

\noindent Sets the name of \textbf{node} to \textbf{name}, freeing the old name
if there was one.\\ Returns 0 on first name, else the old name was freed.

\subsubsection*{AttributeName}
\noindent\begin{tabular}{l l l @{}}
int&\textbf{spConfigAttributeName}&(spConfigAttribute *att,char *name);\\
\end{tabular}

\noindent Sets the name of \textbf{attribute} to \textbf{name}, freeing the old
name if there was one.\\ Returns 0 on first name, else the old name was freed.



\subsection{Loading}


\subsubsection*{Load}
\noindent\begin{tabular}{l l l @{}}
int&\textbf{spConfigLoad}&(spConfigNode *root,char *filename);\\
\end{tabular}

\noindent This function loads a config file from \textbf{filename}, parses it,
and stores whatever it successfully parses in \textbf{root}.\\  Returns 0 on
success, else error.

\subsubsection*{LoadStr}
\noindent\begin{tabular}{l l l @{}}
int&\textbf{spConfigLoadStr}&(spConfigNode *root,char *name,char *string);\\
\end{tabular}

\noindent This function is identical to spConfigLoad above, but instead of a
file it takes a string.  this allows the program to store a compressed and/or
encrypted configuration file, as long as this function gets plain text for
parsing. name is the name that will show up in messages as the file name.\\
Returns 0 on success, else error.



\subsection{Saving}

\subsubsection*{Save}
\noindent\begin{tabular}{l l l @{}}
int&\textbf{spConfigSave}&(spConfigNode *root,char *filename);\\
\end{tabular}

\noindent This function saves a configtree based on \textbf{root} in a file
named \textbf{filename}. This file can then be opened and reparsed with
spConfigLoad() to return an identical tree at a later time.\\ Returns 0 on
success, else error.

\subsubsection*{SaveStr (NOT YET IMPLEMENTED)}
\noindent\begin{tabular}{l l l @{}}
char&\textbf{*spConfigSaveStr}&(spConfigNode *root);\\
\end{tabular}

\noindent This function returns a string representing the config file in
parsable form. instead of saving the config tree in a file, this allows the
program to compress and/or encrypt the config string before saving, effectively
protecting the configuration from casual hacking.




\subsection{Messages}

\subsubsection*{MessagesList}
\noindent\begin{tabular}{l l l @{}}
char&\textbf{*spConfigMessagesList}&(spConfigNode *rootNode);\\
\end{tabular}

\noindent This function returns a string containing all the messages in
\textbf{rootNode}.  This is suitable for printing on the screen or saving to a
file.

\subsubsection*{MessagesClear}
\noindent\begin{tabular}{l l l @{}}
void&\textbf{spConfigMessagesClear}&(spConfigNode*rootNode);\\
\end{tabular}

\noindent This function removes all the messages from \textbf{rootNode} and
reclaims the memory used by those messages.



\subsection{Finding Information}

\subsubsection*{Find}
\noindent\begin{tabular}{l l l @{}}
spConfigNode&\textbf{*spConfigNodeFind}\\
&(spConfigNode *root, char *name, char deep, spConfigNode *previous);\\

spConfigAttribute&\textbf{*spConfigAttributeFind}\\
&(spConfigNode *root, char *name, char deep, spConfigAttribute *previous);\\
\end{tabular}

\noindent These functions search through the entire config tree specified by
\textbf{root}, looking for nodes or attributes with the name \textbf{name}.  If
\textbf{deep} is non-zero, child nodes of \textbf{root} are recursively
included in the search.  If \textbf{previous} is non-null, it is assumed to be
a pointer to the previous node/attribute returned, and the function will return
the next node/attribute after that one.  On the first call, \textbf{previous}
should be NULL.

\subsubsection*{FindPath}
\noindent\begin{tabular}{l l l @{}}
spConfigNode&\textbf{*spConfigNodeFindPath}&(spConfigNode *root,char *path);\\
spConfigAttribute&\textbf{*spConfigAttributeFindPath}&(spConfigNode *root,char *path);\\
\end{tabular}

\noindent These functions return a node or attribute pointing to the node or
attribute indicated in \textbf{path}.  \textbf{path} is a string similar to a
unix path.  NULL is returned  if nothing is found.  If some of the
\textbf{path} is correct, it returns the closest match  (deepest matching
directory).

\subsubsection*{GetPath}
\noindent\begin{tabular}{l l l @{}}
char&\textbf{*spConfigNodeGetPath}&(spConfigNode *node);\\
\end{tabular}

\noindent This function returns the path string of \textbf{node}.\\



\subsection{Debugging}

For detailed information regarding debugging, please see {\tt doc/DEBUGGING}

\subsubsection*{File Output}
\noindent\begin{tabular}{l l l @{}}
void&\textbf{DEBUGFD\_spconfig}&(FILE *debuglog);\\
\end{tabular}

\noindent This is a debugging function used to specify where the debug messages
should go. \textbf{debuglog} should be a file opened for writing.  The default
log is stderr. \textbf{This function has no effect if spconfig is compiled
without debugging.}

\subsubsection*{Debug Level}
\noindent\begin{tabular}{l l l @{}}
void&\textbf{DEBUGLEVEL\_spconfig}&(signed char level);\\
\end{tabular}

\noindent This is a debugging function used to specify the verbosity of the
debug messages.  \textbf{level} is a value between -128 and 127.  Lower levels
result in more messages.  The default level is 0. \textbf{This function has no
effect if spconfig is compiled without debugging.}

\end{document}
